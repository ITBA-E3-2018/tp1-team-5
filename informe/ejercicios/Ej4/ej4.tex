\section{Ejercicio 4}

\subsection{Introducción}
\hspace{10mm} Para implementar un circuito que convierta un número binario de 4 bits en su complemento a dos se empezó pensando este circuito como una caja negra con 4 entradas y 4 salidas:
\begin{figure}[H]
    \centering
	\includegraphics[scale=0.4]{comp2.png}
	\caption{Caja Negra Complemento a 2}
\end{figure}
\hspace{10mm} 
A su vez, sabemos que el complemento a dos se realiza aplicando el complemento a uno y luego sumando uno al resultado. Por lo que se puede representar mediante el siguiente conjunto de "cajas negras":
\begin{figure}[H]
    \centering
	\includegraphics[scale=0.4]{comp1+1.png}
	\caption{Caja Negra Complemento a 2}
\end{figure}
\hspace{10mm} La salida tras el complemento a uno sabemos que es cada bit negado, es decir que atraviesan una compuerta NOT. De esta manera a la salida del complemento a uno que se obtiene es: \\
\centerline{$\overline{a} \overline{b} \overline{c} \overline{d}$} 
\hspace{10mm} Ahora lo único que falta es sumarle uno.

\subsection{Expresión de la salida en mintérminos}
\hspace{10mm} La expresión de la salida queda determinada por las entradas a través de la siguiente fórmula deducida en la introducción:\\
\centerline{$\overline{a} \overline{b} \overline{c} \overline{d}$}
\centerline{$+0001$}
\centerline{$----$}
\centerline{$efgh$}	
\hspace{10mm} Para la creación de las tablas de valor se debe observar que el output de cada bit depende del resultado de el input de los bits menos significativos que con el que se está trabajando.
Para el BMS la tabla de verdad es muy simple y no vale la pena el uso de mintérminos por lo que se verá a continuación:
\begin{table}[H]
	\begin{center}
		\caption{Tabla para bit h}
		\vspace{5mm}
		\begin{tabular}{l|c|r}
			\textbf{$\overline{d}$} & \textbf{1} & \textbf{h} \\
			\hline
			1                       & 1          & 0          \\
			0                       & 1          & 1          \\			
		\end{tabular}
	\end{center}
\end{table}

Fácilmente de la tabla se puede observar que h resulta ser la entrada negada. Por lo que resulta la primera relación de entrada-salida:\\
\begin{equation}
	\centerline{$h(d)=\overline{\overline{d}}$}
\end{equation}
Para la salida correspondiente al bit g la tabla de verdad es de la siguiente manera:
\begin{table}[H]
	\begin{center}
		\caption{Tabla para bit g}
		\vspace{5mm}
		\begin{tabular}{l|c|r}
			\textbf{$\overline{c}$} & \textbf{$\overline{d}$} & \textbf{g}  \\
			\hline
			0                       & 0                       & 0           \\
			0                       & 1                       & 1 ($m_{1}$) \\	
			1                       & 0                       & 1 ($m_{2}$) \\
			1                       & 1                       & 0           \\			
		\end{tabular}
	\end{center}
\end{table}	
Al escribir la salida en función de los mintérminos se llega a la siguiente expresión:\\
\centerline{$g(c,d)=m_{1}+m_{2}$}
\begin{equation}
	\centerline{$g(c,d)=\overline{\overline{c}}\overline{d}+\overline{c}\overline{\overline{d}}$}
\end{equation}
Ahora para el bit f se requiere una tabla de verdad de tres variables ya que ahora depende del carry causado por las variables de entrada c y d:\\
\begin{table}[H]
	\begin{center}
		\caption{Tabla para bit f}
		\vspace{5mm}
		\begin{tabular}{l|c|c|r}
			\textbf{$\overline{b}$} & \textbf{$\overline{c}$} & \textbf{$\overline{d}$} & \textbf{f} \\
			\hline
			0                       & 0                       & 0                       & 0          \\
			0                       & 0                       & 1                       & 0          \\	
			0                       & 1                       & 0                       & 0          \\
			0                       & 1                       & 1                       & 1($m_{3}$) \\
			1                       & 0                       & 0                       & 1($m_{4}$) \\
			1                       & 0                       & 1                       & 1($m_{5}$) \\	
			1                       & 1                       & 0                       & 1($m_{6}$) \\
			1                       & 1                       & 1                       & 0          \\				
		\end{tabular}
	\end{center}
\end{table}
Al escribir la salida en función de los mintérminos se llega a la siguiente expresión:\\
\centerline{$f(b,c,d)=m_{3}+m_{4}+m_{5}+m_{6}$}
\begin{equation}
	\centerline{$f(b,c,d)=\overline{\overline{b}}\overline{c}\overline{d}+\overline{b}\overline{\overline{c}}\overline{\overline{d}}+\overline{b}\overline{\overline{c}}\overline{d}+\overline{b}\overline{c}\overline{\overline{d}}$}
\end{equation}

Para el bit e la tabla de verdad es la siguiente:\\
\begin{table}[H]
	\begin{center}
		\caption{Tabla para bit e}
		\vspace{5mm}
		\begin{tabular}{l|c|c|c|r}
			\textbf{$\overline{a}$} & \textbf{$\overline{b}$} & \textbf{$\overline{c}$} & \textbf{$\overline{d}$} & \textbf{e}  \\
			\hline
			0                       & 0                       & 0                       & 0                       & 0           \\
			0                       & 0                       & 0                       & 1                       & 0           \\	
			0                       & 0                       & 1                       & 0                       & 0           \\
			0                       & 0                       & 1                       & 1                       & 0           \\
			0                       & 1                       & 0                       & 0                       & 0           \\
			0                       & 1                       & 0                       & 1                       & 0           \\	
			0                       & 1                       & 1                       & 0                       & 1($m_{6}$)  \\
			0                       & 1                       & 1                       & 1                       & 1($m_{7}$)  \\		
			1                       & 0                       & 0                       & 0                       & 1($m_{8}$)  \\
			1                       & 0                       & 0                       & 1                       & 1($m_{9}$)  \\	
			1                       & 0                       & 1                       & 0                       & 1($m_{10}$) \\
			1                       & 0                       & 1                       & 1                       & 1($m_{11}$) \\
			1                       & 1                       & 0                       & 0                       & 1($m_{12}$) \\
			1                       & 1                       & 0                       & 1                       & 1($m_{13}$) \\	
			1                       & 1                       & 1                       & 0                       & 1($m_{14}$) \\
			1                       & 1                       & 1                       & 1                       & 0           \\			
		\end{tabular}
	\end{center}
\end{table}
Al escribir la salida en función de los mintérminos se llega a la siguiente expresión:\\
\centerline{$e(a,b,c,d)=m_{6}+m_{7}+m_{8}+m_{9}+m_{10}+m_{11}+m_{12}+m_{13}+m_{14}$}
\begin{equation}
	\centerline{$e(a,b,c,d)=\overline{\overline{a}}\overline{b}\overline{c}\overline{d}+\overline{a}\overline{\overline{b}}\overline{\overline{c}}\overline{\overline{d}}+\overline{a}\overline{\overline{b}}\overline{\overline{c}}\overline{d}+\overline{a}\overline{\overline{b}}\overline{c}\overline{\overline{d}}+\overline{a}\overline{\overline{b}}\overline{c}\overline{d}+\overline{a}\overline{b}\overline{\overline{c}}\overline{\overline{d}}+\overline{a}\overline{b}\overline{\overline{c}}\overline{d}+\overline{a}\overline{b}\overline{c}\overline{\overline{d}}$}
\end{equation}


\subsection{Simplificación de las ecuaciones resultantes}
Para reducir las 4 expresiones resultantes procedemos a utilizar la propiedad $~~\overline{\overline{a}}=a$.\\
Lo cual da como resultado las 4 siguientes ecuaciones:
\begin{align*}
	h & = d                                                                                                                                                                                                                                                        \\
	g & = c\overline{d}+\overline{c}d                                                                                                                                                                                                                              \\
	f & = b\overline{c}\overline{d}+\overline{b}cd+\overline{b}c\overline{d}+\overline{b}\overline{c}d                                                                                                                                                             \\
	e & = a\overline{b}\overline{c}\overline{d}+\overline{a}bcd+\overline{a}bc\overline{d}+\overline{a}b\overline{c}d+\overline{a}b\overline{c}\overline{d}+\overline{a}\overline{b}cd+\overline{a}\overline{b}c\overline{d}+\overline{a}\overline{b}\overline{c}d 
\end{align*}
De estas cuatro expresiones las primeras son irreducibles. Pero es posible reducir la tercera y la cuarta con el uso de las siguientes propiedades:
\begin{align*}
	a+a            & = a      \\
	a+\overline{a} & = 1      \\
	ab+cb          & = b(a+c) 
\end{align*} 
La primera propiedad significa que puedo sumar uno de los mintérminos a la expresión que esta no se verá afectada si ese mintérmino ya pertenecía a la expresión. Con la segunda expresión puedo sacar 'factor común' los términos que difieran en una variable y su complemento de manera que al hacer el factor común quede de la forma $f(a,b,...)(z+\overline{z})$ y así utilizar la tercera propiedad que simplifica esa expresión a $f(a,b,...)$. De esta manera las ecuaciones se simplifican a su forma final que es de la siguiente manera:
\begin{align*}
	h & = d                                                                                                       \\
	g & = c\overline{d}+\overline{c}d                                                                             \\
	f & = b\overline{c}\overline{d}+\overline{b}c+\overline{b}d                                                   \\
	e & = a\overline{b}\overline{c}\overline{d}+\overline{a}\overline{c}d+\overline{a}\overline{b}c+\overline{a}b 
\end{align*}


\subsection{Circuito Resultante}
Al implementar las expresiones halladas en la sección anterior con compuertas AND, OR y NOT se llega al siguiente circuito: \\

\begin{figure}[H]
    \begin{center}
	\includegraphics[scale=0.6]{circuitoEj4.png}
    \caption{Circuito de conversión a complemento a dos}
    \end{center}
\end{figure}