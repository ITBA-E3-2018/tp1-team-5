\documentclass{article}
\usepackage{graphicx}
\usepackage{amsmath}
\usepackage{geometry}
 \geometry{
 a4paper,
 total={170mm,257mm},
 left=20mm,
 top=20mm,
 }
 
\begin{document}
	\pagenumbering{arabic}
	Comentarios:
	\newline -Falta agregar tildes
	\section{Ejercicio 4}
	\subsection{Introduccion}
		\hspace{10mm} Para implementar un circuito que convierta un numero binario de 4 bits en su complemento a dos se empezo pensando este circuito como una caja negra con 4 entradas y 4 salidas:
		\begin{figure}[h!]
			\includegraphics[width=\linewidth,scale=0.5]{comp2.png}
  			\caption{Caja Negra Complemento a 2}
		\end{figure}
		\newline \hspace{10mm} A su vez, sabemos que el complemento a dos se realiza aplicando el complemento a uno y luego sumando uno al resultado. Por lo que se puede representar mediante el siguiente conjunto de "cajas negras":
		\begin{figure}[h!]
			\includegraphics[width=\linewidth,scale=0.5]{comp1+1.png}
  			\caption{Caja Negra Complemento a 2}
		\end{figure}
		\newline \hspace{10mm} La salida tras el complemento a uno sabemos que es cada bit negado, es decir que atraviesan una compuerta NOT. De esta manera a la salida de el complemento a uno que se obtiene es: 
		\newline \centerline{$\overline{a} \overline{b} \overline{c} \overline{d}$} 
		\hspace{10mm} Ahora lo unico que falta es sumarle uno.
	\subsection{Expresion de la salida en minterminos}
		\hspace{10mm} La expresion de la salida queda determinada por las entrada a traves de la siguiente formula deducida en la introduccion:
		\newline \centerline{$\overline{a} \overline{b} \overline{c} \overline{d}$}
		\newline \centerline{$+0001$}
		\newline \centerline{$----$}
		\newline \centerline{$efgh$}	
		\newline \hspace{10mm} Para la creacion de las tablas de valor se debe observar que el output de cada bit depende de el resultado de el input de los bits menos significativos que con el que se esta trabajando.
		\newline Para el BMS la tabla de verdad es muy simple y no vale la pena el uso de minterminos por lo que se vera a continuacion:
		\begin{table}[h!]
			\begin{center}
				\caption{Tabla para bit h}
				\begin{tabular}{l|c|r}
				\textbf{$\overline{d}$} & \textbf{1} & \textbf{h}\\
				\hline
				1 & 1 & 0 \\
				0 & 1 & 1 \\			
				\end{tabular}
			\end{center}
		\end{table}
		\newline Facilmente de la tabla se puede observar que h resulta ser la entrada negada. Por lo que resulta la primera relacion de entrada-salida:
		\newline \centerline{$h(d)=\overline{\overline{d}}$}
		\newline Para la salida correspondiente a el bit g la tabla de verdad es de la siguiente manera:
		\begin{table}[h!]
			\begin{center}
				\caption{Tabla para bit h}
				\begin{tabular}{l|c|r}
				\textbf{$\overline{c}$} & \textbf{$\overline{d}$} & \textbf{g}\\
				\hline
				0 & 0 & 0 \\
				0 & 1 & 1 ($m_{1}$) \\	
				1 & 0 & 1 ($m_{2}$)\\
				1 & 1 & 0 \\			
				\end{tabular}
			\end{center}
		\end{table}	
		\newline Al escribir la salida en funcion de los minterminos se llega a la siguiente expresion:
		\newline \centerline{$g(c,d)=m_{1}+m_{2}$}
		\newline \centerline{$g(c,d)=\overline{\overline{c}}\overline{d}+\overline{c}\overline{\overline{d}}$}
\end{document}